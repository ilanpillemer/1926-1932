\documentclass{article}
%\usepackage[margin=2in]{geometry}
\usepackage[osf,p]{libertinus}
\usepackage{microtype}
\usepackage[pdfusetitle,hidelinks]{hyperref}

\usepackage[series={A,B,C}]{reledmac}
\usepackage{reledpar}

\usepackage{graphicx}
\usepackage{polyglossia}
\setmainlanguage{english}
\setotherlanguage{hebrew}
\gappto\captionshebrew{\renewcommand\chaptername{קאַפּיטל}}
\usepackage{metalogo}


%%linenumincrement*{1}
%%\firstlinenum*{15}
%%\setlength{\Lcolwidth}{0.44\textwidth}
%%\setlength{\Rcolwidth}{0.44\textwidth}

\begin{document}
%%\maxhnotesA{0.8\textheight}
\renewcommand{\abstractname}{\vspace{-\baselineskip}}
\title{1926, 1932}
\author{Extracts from a Warsaw Yizker Buch \\ Transl. Ilan Pillemer}
\date{\today}

\maketitle

\section{ 1916  }
\begin{itemize}
\item Founding of the Orthodox Agudah Yisrael Party, Folkspartey.
\item Founding of literature and journalistic bodies.
\item A significant amount of  literature and journalistic activity of many kinds.
\end{itemize}

\section{ 1918 }
\begin{itemize}
\item Warsaw is liberated.
\item Many people come to Warsaw looking for opportunity.
\item There are deportations, which include Jews.
\item The Jewish board is active quickly in such a dynamic time.
\end{itemize}
\section{ 1919 }
\begin{itemize}
\item There is a big influx of Galizianers.
\item Luzian Peretz, the only son of YL Peretz, passes away, leaving only Yanka Peretz as the last descendant.
\item There is free political activity, with many Jews of different political stripes.
\end{itemize}

\section { 1921 }
\begin{itemize}
\item Founding of workers parties.
\item Founding of Hebrew cultural activity.
\item Jews makes up about one third of the population.
\item Jews are very involved in industrialisation projects.
\item Professor Moshe Shor becomes the preacher at the main synagogue.
\end{itemize}
\begin{pairs}

\begin{Rightside}

\begin{RTL}
\begin{hebrew}
\firstlinenumR{10000000}
\beginnumbering
%%\numberpstarttrue

\autopar

דאָס ײדישע לעבן אין װאַרשע צװישן בײדע װעלט־מלחמות גײט „נאָרמאַל“
- קריזיס־אײן - קריזיס־אוים - מיט פּאַליטישע און עקאָנאָמישע קאַמפֿן
און אַ כסדר װאַקסניקער קולטורעלער רענעסאַנס־טעטיקײט.
די קהילה װערט דעמאָקראַטיזירט. אויף די זיצונגען קען מען רעדן: ײדיש, העבריעש, פּויליש.
דער צוזאַמענשטעל: 18 ציונים, 16 אגודה, 4 אָרטאָדאָקסן אומפּאַרטײאישע, 6 בונדיסטן, 3 פֿאָלקיסטן, 2 לינקע פּועלי ציון.
העשל פֿאַרבשטײן - ראש פֿון דער קהילה ביזון יאָר 1931. גרינדונג פֿון „װיק“־טעאַטער, דירעקציע זיגמונט טורקאָװ.

\endnumbering
\end{hebrew}
\end{RTL}
\end{Rightside}


\begin{Leftside}
\begin{english}
\section{ 1926  }
\beginnumbering
\autopar

The Jewish life in Warsaw between the two world wars passed "normally" - a crisis comes, a crisis goes\footnoteA{
Same phrasing as a "Year in, Year out" i.e. \RLE{“יאָר אײן, יאָר אויס„}
} with political and economic struggles and constantly growing cultural renaissance activity.  
The community assembly was democratic. At their meetings one could hear spoken: Yiddish,
Hebrew and Polish.
Altogether they were made up of: 18 Zionist, 16 Agudah, 4 a-political Orthodox, 6 Bundist, 3 Folkspartey 
and 2 left-wing Poalei Tzion.
Herschel Farbstein - was the head of the community until 1931. 
The founding of the "Vikt" theatre managed by Zigmund Turkav.

\endnumbering
\end{english}
\end{Leftside}
\end{pairs}
\Columns

\begin{pairs}

\begin{Rightside}

\begin{RTL}
\begin{hebrew}
\firstlinenumR{10000000}
\beginnumbering
%%\numberpstarttrue

\autopar

נויט און אײביקע אומזיכערקײט מיטן מאָרגן, דערצו די הערמעטיש געשלאָסענע
גרענצן פֿון די אימיגראַציע־לענדער, טרײבן צו העכסטער פֿאַרצװײפֿלונג די ײדישע מאַסן אין װאַרשע.
אַ כאַראַקטעריסטישער עפּיזאָד:
אַ ײדישער זשורנאַליסט, װאָס האָט באַזוכט רוסלאַנד, טײלט מיט, אַז עס זײנען דאָ
עמיגראַציע־מעגלעכקײטן קײן ביראָ־בידזשאַן. דאָרט און דאָרט זיך מעלדן.
אין אײן אײנציקן טאָג - האָבן אַן ערד צען טויזנט בעלנים אין װאַרשע זיך געמאָלדן. פֿוּן פּראָיעקט איז גאָרנישט געװאָרן.

\endnumbering
\end{hebrew}
\end{RTL}
\end{Rightside}


\begin{Leftside}
\begin{english}
\section{ 1932  }
\beginnumbering
\autopar

Need and constant uncertainty about the future, in addition to the hermetic locked borders of the immigration-countries, 
drove the Jewish masses in Warsaw to utmost despair. A typical event: a Jewish journalist, whilst visiting Russia,
reports that there are immigration opportunities  to Birobidzhan. Over and over there were reports. On one particular
day there could be imagined approximately ten thousand applications. There were no projects.

\endnumbering
\end{english}
\end{Leftside}
\end{pairs}
\Columns


\end{document}



















































